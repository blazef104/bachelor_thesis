\documentclass[../main.tex]{subfiles}
\begin{document}
\chapter{Avionics Protocols}
\label{chap:avionicsprot}
\todo{explain the main avionics protocols that will be discussed in the research. }
Avionics telecomunications have undergone a major upgrade in the last years which is still in progress and new standards are set to became operative in the next decade. Such changes are made to bring the benefits of modern tecnology onboard of the aircrafts and in the \textbf{Air Traffic Management} (\textbf{ATM}) world. The classic radar will progressively be sobstituted by other systems based on transponders which require an active partecipation of the aircraft in the tracking procedure, meanwhile the classic radar will be kept as secondary or backup tracking method. Having an active data connection between the ground and the aircraft has many benefits, the most important is the ability to have a bilateral comunication where the ground can receive information on the status of the aircraft and of the flight and the aircraft can receive updated information on the weather, congestions and so on. All this informations, the majority of which are handled automatically by the systems, helps lowering the workload of both the pilots and the controllers while helping and enhancing both jobs.
\todo{acars messages}

\section{OldGen}
\todo{explain how 1090mhz protocols works}


\section{NextGen}
\todo{explain how 978/nexrad/tis-b/fis-b protocols works}

\todo{maybe explain backwards compatibility with 1090 using 1090ES (extendes squitter) also put the theory about using fis-b tis-b in 1090mhz band. (Find some references and obtain some documents FREE)}

\end{document}
