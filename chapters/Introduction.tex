\documentclass[../main.tex]{subfiles}
\begin{document}
\chapter{Introduction}

Over these last 15 years the introduction of more and more automated systems on board of aircrafts and inside the Air Traffic Control and Air Traffic Management has seen an incredibly fast growth compared to the past. Speaking of the General Aviation sector, which includes all the non-military aircrafts different from scheduled air services and non scheduled salaried air transport, this transformation is even more evident. In particular more and more systems are designed as IoT devices with a strong decision-making autonomy, the Garmin G1000\footnote{\url{https://buy.garmin.com/en-US/US/p/6420}} is a perfect example as it is a Glass Flight Deck which integrates all the avionics and can also be connected with an Electronic Flight Bag that lets the pilot perform some operations directly on an Ipad.

Since new air carriers must undergo long processes of research and development which take decades before a "new generation" airplane can go into service, the avionics must at some extent follow this timing and therefore a too rapid or sudden development, in the commercial aviation, is not possible.

New avionics protocols, having been designed with an IoT paradigm in mind, are now deeply integrated inside organization networks and are at the base of services like Flightradar24. Moreover, such protocols are used, not only by aircrafts, but also by drones, gliders and similar; while it is true that this enhances the security of all the flying objects and allows almost all of them to be connected, on the other hand a possible vulnerability inside those protocols can lead to dangerous consequences.

The protocols examined are: \textbf{ADS-B}, \textbf{FIS-B}, \textbf{ModeS} that are messaging protocols and \textbf{1090ES}, \textbf{UAT} which are lower level transport protocols. Previous researches had already demonstrated the insecurity of such protocols in different ways and more recently the Department of Homeland Security successfully conducted a non-cooperative RF hack on a Boeing 757 which gave them complete control over the aircraft.

The innovative approach of my research is the use of Fuzzing to test avionics protocols. This is a software testing method based on the generation of completely random or mutated inputs to trigger crashes in the target binary. Such method has already been proven effectively in finding important bugs such as Heartbleed, OSX Kernel memory corruptions, libjpgturbo corruptions and many more. The specific tool used during the research is American Fuzzy Lop, a feature-rich and open source fuzzer developed inside Google Project Zero initiative. A particular fork of this tool, \textit{afl-unicorn} has been used as it was developed specifically to deep test programs interacting with Radio Frequencies and it allows to selectively emulate and test programs for a different architecture.
The very nature of such programs, which are specifically designed to be noise resistant, employ different strategies, which bring some challenges in the Fuzzing attempt, in order to correct possible errors.

This work has a dual importance because the tests are conducted on open source implementations of such protocols, which allows to test the IoT device as well as the protocols themselves. Although the main focus was on finding vulnerabilities on the protocol side a potential presence of a software specific vulnerability might give hints on most problematic points. Moreover, if a vulnerability is present in the open source software there is a high possibility that it would be present in the commercial version, nevertheless a vulnerability might also be present only on the commercial side and vice versa.

The fast growing progress in the avionics instruments and related protocols must go along with security researches; therefore it is paramount to develop knowledge and tools that can help discover vulnerabilities in \textbf{NextGen} and avionics as easy and as early as possible.


\bigskip \noindent
The main body of my dissertation is organized as follows:

\bigskip \noindent
In \textit{Chapter \ref{chap:avionicsprot}: \nameref{chap:avionicsprot}} there is an overview of the present situation and an analysis of the current state of the art of the avionics protocols. A clear description of the two families of protocols, \textbf{OldGen} and \textbf{NextGen}, is given in order to show the critical points of those protocols. While there is an overview of the most relevant avionics protocols, particular attention is given to the protocols analyzed during my research.


\bigskip \noindent
In \textit{Chapter \ref{chap:fuzzing}: \nameref{chap:fuzzing}} some basic information on the most common software testing methods is given with particular focus on the fuzzing. Two of the state of the art tools in term of binary fuzzing are then described: \textbf{AFL} and \textbf{Peach}. I focused on \textbf{AFL}, since this is the tool used during the research, giving detailed information on the techniques used by this fuzzer and on how to read its interface. Moreover a description of a modified version of \textbf{AFL}: \textit{afl-unicorn} is provided.

\bigskip \noindent
In \textit{Chapter \ref{chap:experimsetup}: \nameref{chap:experimsetup}} the experimental setup of the research is illustrated, dividing the chapter into \textbf{Hardware Setup} and \textbf{Software Setup}. The first part is about the hardware used while the second one describes the binaries tested, the datasets used and how they have been acquired and, lastly, the software and scripts specifically created for this research.

\bigskip \noindent
In \textit{Chapter \ref{chap:resconc}: \nameref{chap:resconc}} the results and caveats encountered during the research are described. In particular detailing the problems that have been encountered with the flavoured version of \textbf{AFL}: \textit{afl-unicorn} and with the underlaying emulator, the Unicorn Engine. Also there are some considerations on the datasets and on the programs tested.

\bigskip \noindent
In \textit{Chapter \ref{chap:futurework}: \nameref{chap:futurework}} there are the conclusions and an overall analysis of what has been done. More importantly some ideas and proposals to further continue and expand this work are discussed.
\newline
%\item \textit{Chapter 7: }\newline

\end{document}
